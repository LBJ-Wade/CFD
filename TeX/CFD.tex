\documentclass[10pt]{article}
\usepackage[english]{babel}
\usepackage{amsmath}
\usepackage{color}
\usepackage{amsfonts}
\usepackage{xcolor,graphicx}
\usepackage{geometry}
\usepackage{animate}
\usepackage{listings}
\usepackage{wrapfig}
\usepackage{subcaption}

\usepackage{lipsum}

%Modified from:
%http://mathematica.stackexchange.com/questions/42793/highlighting-mathematica-code-in-latex-document

\definecolor{listinggray}{gray}{0.9}
\definecolor{graphgray}{gray}{0.7}
\definecolor{blue}{rgb}{0,0,1}
\definecolor{mygreen}{rgb}{0,0.6,0}
\definecolor{mygray}{rgb}{0.5,0.5,0.5}
\definecolor{purple}{rgb}{0.5,0,0.5}
\definecolor{orange}{rgb}{1,0.5,0}

% define a custom mathematica language for syntax highlighting
\lstdefinelanguage{myMMA}{
keywords={SetDirectory, ColorData, BarLegend, LegendLabel, Graphics, Gray, Polygon, Import, Do, If, Dimensions, Break, AppendTo, Sqrt, All, Length, ListDensityPlot, ColorFunction, PlotRange, AspectRatio, PlotLegends, ColorFunctionScaling, FrameLabel, PlotLabel, Style, FontSize, Black, ImageSize, Full, Show, Automatic, False, LegendMarkerSize, ToString, $CommandLine, Print, LabelStyle, ListContourPlot, ContourShading, Export, Contours},
keywords=[2]{i},
keywordstyle=\color{black},
keywordstyle=[2]\color{mygreen},
commentstyle=\color{gray}, 
stringstyle=\color{purple},
identifierstyle=\color{blue},
sensitive=false,
comment=[l]{(*},
morecomment=[s]{/*}{*/},
morecomment=[l][\color{orange}]{\#!},
morestring=[b]',
morestring=[b]",
breaklines=true,
captionpos=b,
numbers=left,
literate={->}{$\rightarrow{}$}{1}
}


\title{Computational Fluid Dynamics}
\author{Brady Metherall \and 100516905}
\date{Monday April 4, 2016}

\begin{document}
\newgeometry{margin=1in}
\maketitle
\setlength\parindent{0pt}
\lstset{language=myMMA}

\section{Introduction}

\begin{wrapfigure}{r}{0.5\textwidth}
\label{airfoilanimation}
\centering
\animategraphics[label=airfoil, loop, width = 0.5\textwidth, autoplay]{12}{Airfoil_Animation_Turb/Square_Cylinder}{0050}{0074}
\caption{Look at how neat that is!}
\end{wrapfigure}

\lipsum[1-4]

\section{Theory}

\section{SU2}

\subsection{Mesh}

\section{Code}

\begin{figure}[p]
\lstinputlisting[breaklines=true, label=makecontour, caption={The Wolfram script used to generate Figure \ref{airfoilanimation}.}]{MakeContour.m}
\end{figure}

\section{Results}

\begin{figure}[ht]
\centering
\label{boundary}
\includegraphics[width=0.75\textwidth]{./Pictures/Static_Cylinder_NS_Boundary}
\caption{The boundary layer that forms due to a viscous fluid.}
\end{figure}

\begin{figure}
\begin{subfigure}{\linewidth}
  \centering
  \includegraphics[height=0.17\paperheight]{./Pictures/Vortex_Shedding1400}
  \caption{$t = 0$}
\end{subfigure} \\
\begin{subfigure}{\textwidth}
  \centering
  \includegraphics[height=0.17\paperheight]{./Pictures/Vortex_Shedding1427}
  \caption{$t = \frac{1}{4}T$}
\end{subfigure} \\
\begin{subfigure}{\linewidth}
  \centering
  \includegraphics[height=0.17\paperheight]{./Pictures/Vortex_Shedding1454}
  \caption{$t = \frac{1}{2}T$}
\end{subfigure} \\
\begin{subfigure}{\textwidth}
  \centering
  \includegraphics[height=0.17\paperheight]{./Pictures/Vortex_Shedding1481}
  \caption{$t = \frac{3}{4}T$}
\end{subfigure}
\caption{Vortex shedding!}
\label{fig:vortexshedding}
\end{figure}



\end{document}