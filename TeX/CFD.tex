\documentclass[10pt]{article}
\usepackage[english]{babel}
\usepackage{amsmath}
\usepackage{color}
\usepackage{amsfonts}
\usepackage{xcolor,graphicx}
\usepackage{geometry}
\usepackage{animate}
\usepackage{listings}
\usepackage{wrapfig}
\usepackage{subcaption}
\usepackage{hyperref}

\usepackage{lipsum}

%Modified from:
%http://mathematica.stackexchange.com/a/83811

\definecolor{listinggray}{gray}{0.9}
\definecolor{graphgray}{gray}{0.7}
\definecolor{blue}{rgb}{0,0,1}
\definecolor{mygreen}{rgb}{0,0.6,0}
\definecolor{mygray}{rgb}{0.5,0.5,0.5}
\definecolor{purple}{rgb}{0.5,0,0.5}
\definecolor{orange}{rgb}{1,0.5,0}

% define a custom mathematica language for syntax highlighting
\lstdefinelanguage{myMMA}{
keywords={SetDirectory, ColorData, BarLegend, LegendLabel, Graphics, Gray, Polygon, Import, Do, If, Dimensions, Break, AppendTo, Sqrt, All, Length, ListDensityPlot, ColorFunction, PlotRange, AspectRatio, PlotLegends, ColorFunctionScaling, FrameLabel, PlotLabel, Style, FontSize, Black, ImageSize, Full, Show, Automatic, False, LegendMarkerSize, ToString, $CommandLine, Print, LabelStyle, ListContourPlot, ContourShading, Export, Contours},
keywords=[2]{i},
keywordstyle=\color{black},
keywordstyle=[2]\color{mygreen},
commentstyle=\color{gray}, 
stringstyle=\color{purple},
identifierstyle=\color{blue},
sensitive=false,
comment=[l]{(*},
morecomment=[s]{/*}{*/},
morecomment=[l][\color{orange}]{\#!},
morestring=[b]',
morestring=[b]",
breaklines=true,
captionpos=b,
numbers=left,
literate={->}{$\rightarrow{}$}{1}
}


\title{Computational Fluid Dynamics}
\author{Brady Metherall \and 100516905}
\date{Monday April 4, 2016}

\begin{document}
\newgeometry{margin=1in}
\maketitle
\setlength\parindent{0pt}
\lstset{language=myMMA}

\section{Introduction}

\begin{wrapfigure}{R}{0.5\textwidth}
\centering
\animategraphics[loop, width = 0.5\textwidth, autoplay]{12}{Airfoil_Animation_Turb/Square_Cylinder}{0050}{0074}
\caption[Pitching Airfoil Animation]{Look at how neat that is!}
\label{fig:airfoilanimation}
\end{wrapfigure}

\lipsum[1-4]

\newpage
\listoffigures
\newpage

\section{Theory}
\lipsum[1-2]

\begin{align*}
\frac{D\bf{u}}{Dt} &= \frac{\partial \bf{u}}{\partial t} + \left( \bf{u} \cdot \nabla \right) \bf{u}
\end{align*}

Euler's Equations
\begin{equation}
\label{euler}
\begin{aligned}
\frac{\partial \bf{u}}{\partial t} + \left( \bf{u} \cdot \nabla \right) \bf{u} &= - \frac{1}{\rho} \nabla p + \bf{g} \\
\nabla \cdot \bf{u} &= 0
\end{aligned}
\end{equation}

Cauchy's Equation
\begin{align}
\rho \left( \frac{\partial \bf{u}}{\partial t} + \left( \bf{u} \cdot \nabla \right) \bf{u} \right) &= \nabla \cdot \bf{T} + \rho \bf{g}
\end{align}

Navier-Stokes Equation
\begin{equation}
\begin{aligned}
\rho \left( \frac{\partial \bf{u}}{\partial t} + \left( \bf{u} \cdot \nabla \right) \bf{u} \right) &= - \nabla p + \mu \nabla^2 \bf{u} + \rho \bf{g} \quad \text{Letting } \nu = \frac{\mu}{\rho} \\
\frac{\partial \bf{u}}{\partial t} + \left( \bf{u} \cdot \nabla \right) \bf{u} &= - \frac{1}{\rho} \nabla p + \nu \nabla^2 \bf{u} + \bf{g}
\end{aligned}
\end{equation}


\section{SU2 Code}
\lipsum[1-2]

\subsection{Mesh}
\lipsum[1-2]

\section{Scripting and Automation}
\lipsum[1-2]

\begin{figure}[p]
\lstinputlisting[breaklines=true, label=makecontour]{MakeContour.m}
\caption[Automated Wolfram Script]{The Wolfram script used to generate Figure \ref{fig:airfoilanimation}. Syntax highlighting modified from \cite{syntax}.}
\label{script}
\end{figure}

\section{Results}
\lipsum[1-2]

\subsection{Steady Flow Around a Static Cylinder}
\lipsum[1-2]

\begin{figure}[htb]
\begin{subfigure}{0.5\linewidth}
  \centering
  \includegraphics[width=\textwidth]{./Pictures/Static_Cylinder_Euler}
  \caption{Euler}
  \label{fig:staticeuler}
\end{subfigure}
\begin{subfigure}{0.5\linewidth}
  \centering
  \includegraphics[width=\textwidth]{./Pictures/Static_Cylinder_NS}
  \caption{NS}
  \label{fig:staticNS}
\end{subfigure}
\caption[Static Cylinder]{Static Cylinders}
\label{fig:static}
\end{figure}

\begin{figure}[htb]
\centering
\includegraphics[width=0.75\textwidth]{./Pictures/Static_Cylinder_NS_Boundary}
\caption[Boundary Layer Due to Viscous Flow]{The boundary layer that forms due to a viscous fluid.}
\label{boundary}
\end{figure}

\subsection{Turbulent Cylinder -- Vortex Shedding}

The equation \eqref{euler} is interesting.

\begin{wrapfigure}{R}{0.6\textwidth}
\centering
\animategraphics[label=vortex, loop, width = 0.6\textwidth, autoplay]{12}{Vortex_Animation/Vortex_Shedding}{00}{26}
\caption[Vortex Shedding Animation]{Neat! The full video can be found at \url{https://youtu.be/Zh5lWMpsCJg}}
\label{vortexanimation}
\end{wrapfigure}

\lipsum[1-2]

\begin{figure}
\begin{subfigure}{\linewidth}
  \centering
  \includegraphics[height=0.17\paperheight]{./Pictures/Vortex_Shedding1600}
  \caption{$t = 0$}
\end{subfigure} \\
\begin{subfigure}{\textwidth}
  \centering
  \includegraphics[height=0.17\paperheight]{./Pictures/Vortex_Shedding1628}
  \caption{$t = \frac{1}{4}T$}
\end{subfigure} \\
\begin{subfigure}{\linewidth}
  \centering
  \includegraphics[height=0.17\paperheight]{./Pictures/Vortex_Shedding1656}
  \caption{$t = \frac{1}{2}T$}
\end{subfigure} \\
\begin{subfigure}{\textwidth}
  \centering
  \includegraphics[height=0.17\paperheight]{./Pictures/Vortex_Shedding1684}
  \caption{$t = \frac{3}{4}T$}
\end{subfigure}
\caption[Vortex Shedding at $\frac{1}{4}T$ increments]{Vortex shedding!}
\label{fig:vortexshedding}
\end{figure}

\subsection{Airfoil}
\lipsum[1-2]

\begin{figure}[ht]
\centering
\includegraphics[width=0.75\textwidth]{./Pictures/Airfoil_10_0}
\caption[Airfoil, No Turbulence]{Airfoil no turbulence.}
\label{airfoil}
\end{figure}

\newpage
\begin{thebibliography}{9}

%\bibitem{taketani} %Scale / range
%Taketani, Mituo and Nakamura, Seitaro and Sasaki, Muneo,
%\emph{On the Method of the Theory of Nuclear Forces},
%Progress of Theoretical Physics \textbf{6}, 581-586 (1951).

\bibitem{syntax} %Mathematica syntax highlighting
Belisarius (http://mathematica.stackexchange.com/users/29581/belisarius), Highlighting Mathematica code in \LaTeX document, URL: http://mathematica.stackexchange.com/a/83811

\end{thebibliography}

\end{document}