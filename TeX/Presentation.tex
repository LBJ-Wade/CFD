\documentclass[mathserif]{beamer}
\usepackage[english]{babel}
\usepackage{amsmath}
\usepackage{color}
\usepackage{amsfonts}
\usepackage{xcolor,graphicx}
\usepackage{geometry}
\usepackage{animate}
\usepackage{listings}
\usepackage{subcaption}
\usepackage{hyperref}

\title{Computational Fluid Dynamics}
\author{Brady Metherall \and 100516905}
\date{Thursday April 7, 2016}

\usetheme{Berkeley}
\usecolortheme{whale}

\begin{document}
\frame{\titlepage}
\setlength\parindent{0pt}

\section{Introduction}

\frame{
\frametitle{Introduction}

}

\section{Theory}

\frame{
\frametitle{Non-Turbulent Flow}
\begin{itemize}
\item Incompressibility Condition: $\displaystyle \nabla \cdot \mathbf{u} = 0$

\item Euler's Equations: $\displaystyle \frac{\partial \mathbf{u}}{\partial t} + \left( \mathbf{u} \cdot \nabla \right) \mathbf{u} = - \frac{1}{\rho} \nabla p + \mathbf{g}$

\item Cauchy's Equation: $\displaystyle \rho \left( \frac{\partial \mathbf{u}}{\partial t} + \left( \mathbf{u} \cdot \nabla \right) \mathbf{u} \right) = \nabla \cdot \mathbf{T} + \rho \mathbf{g}$

\item Stress Tensor: $\displaystyle \mathbf{T}_{ij} = -p \delta_{ij} + \mu \left( \frac{\partial u_j}{\partial x_i} + \frac{\partial u_i}{\partial x_j} \right)$

\item Navier-Stokes Equation:
\begin{equation*}\frac{\partial \mathbf{u}}{\partial t} + \left( \mathbf{u} \cdot \nabla \right) \mathbf{u} = - \frac{1}{\rho} \nabla p + \nu \nabla^2 \mathbf{u} + \mathbf{g}
\end{equation*}
\end{itemize}
}

\frame{
\frametitle{Reynolds-Averaged Navier-Stokes}
To make solving the Navier-Stokes equations easier, especially for turbulent flows, we can assume the flow is the superposition of the steady, and turbulent flows, and then take the time average.
\begin{align*}
\alpha_i &\equiv \overline{\alpha_i} + \alpha_i', \\
\text{where } \overline{\alpha_i} &= \frac{1}{\tau} \int_0^\tau \alpha_i dt \\
\overline{\alpha_i'} = 0, \quad \frac{\partial \overline{\alpha_i}}{\partial t} = 0, \quad \overline{\overline{\alpha_i}} &= \overline{\alpha_i}, \quad \text{and} \quad \overline{\frac{\partial \alpha_i}{\partial x_i}} =  \frac{\partial \overline{\alpha_i}}{\partial x_i}
\end{align*}
Making the appropriate substitutions, taking the time average, and using the properties we obtain the Reynolds-Averaged Navier-Stokes equation,
\begin{align*}
\left( \mathbf{\overline{u}} \cdot \nabla \right) \mathbf{\overline{u}} + \overline{\left( \mathbf{u'} \cdot \nabla \right) \mathbf{u'}} = - \frac{1}{\rho} \nabla \overline{p} + \nu \nabla^2 \mathbf{\overline{u}} + \mathbf{\overline{g}} 
\end{align*}
}

\frame{
\frametitle{Spalart-Allmaras Turbulence}
\begin{itemize}
\item A model is still needed for $\nu$
\item According to the Boussinesq hypothesis, an increase in the viscosity gives the effect of turbulence
\item Therefore $\nu = \overline{\nu} + \nu'$
\item One such way to model the turbulence, is the Spalart-Allmaras turbulence model
\end{itemize}
\begin{align*}
\nu' = \hat{\nu} f_{v1}; \qquad f_{v1} = \frac{\chi^3}{\chi^3+c_{v1}^3}; \qquad \chi \equiv \frac{\hat{\nu}}{\overline{\nu}}.
\label{eq:saeq}
\end{align*}
Assuming the flow is incompressible, and the diffusivity is constant, $\hat{\nu}$ is obtained by solving the convection-diffusion equation,
\begin{align*}
\frac{\partial \hat{\nu}}{\partial t} = D \nabla^2 \hat{\nu} - \overline{\mathbf{u}} \nabla \hat{\nu},
\end{align*}
where $D$ is the diffusivity.
}

\section{$SU^2$ Code}

\frame{
\frametitle{$SU^2$ Code}

}

\frame{
\frametitle{Mesh and Numerics}

}

\section{Scripting and Automation}

\frame{
\frametitle{Scripting and Automation}
\begin{itemize}
\item Wolfram Mathematica 10.1.0 for Linux x86 was used to sift through the vast data files and extract the relevant information to produce the plots
\item A Wolfram function was needed to convert the data to a form Mathematica could use
\item Mathematica's \texttt{ListDensityPlot} function was used to create the images
\item The function to tidy the data along with the plotting function were combined into a Wolfram script which can be executed from the terminal
\item  To fully automate the image generation, a shell script was written to iterate over each data file
\end{itemize}
}

\section{Results}

\frame[shrink]{
\frametitle{Airfoil}
\begin{figure}[htb]
\centering
\includegraphics{./Pictures/Airfoil}
\end{figure}
}

\frame[shrink]{
\frametitle{Static Cylinder}
\framesubtitle{Euler}
\begin{figure}[htb]
\centering
\includegraphics{./Pictures/Static_Cylinder}
\end{figure}
}

\frame[shrink]{
\frametitle{Static Cylinder}
\framesubtitle{Navier-Stokes}
\begin{figure}[htb]
\centering
\includegraphics{./Pictures/Static_Cylinder_NS}
\end{figure}
}

\frame[shrink]{
\frametitle{Static Cylinder}
\framesubtitle{Euler Boundary}
\begin{figure}[htb]
\centering
\includegraphics{./Pictures/Static_Cylinder_Boundary}
\end{figure}
}

\frame[shrink]{
\frametitle{Static Cylinder}
\framesubtitle{Navier-Stokes Boundary}
\begin{figure}[htb]
\centering
\includegraphics{./Pictures/Static_Cylinder_Boundary_NS}
\end{figure}
}

\frame{
\frametitle{Vortex Shedding}
\begin{figure}[htb]
\centering
\animategraphics[label=vortex, loop, width = \textwidth, autoplay]{12}{Vortex_Animation/Vortex_Shedding}{00}{26}
\end{figure}
}

\section{Conclusion}

\frame{
\frametitle{Conclusion}

}

\end{document}